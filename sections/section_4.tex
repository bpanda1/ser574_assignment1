impact of software design and architecture teams working in agile

https://www.agileconnection.com/article/agile-approach-software-architecture


In agile, the architecture of a software is developed in an iterative manner, beginning with enough definition for the team to start design and development. The architecture guides the team in a direction that gives the best chance for success.

There is a trade-off between allowing a team to choose what makes sense for their project while ensuring they don’t pick technologies that are incompatible with the rest of the enterprise. This trade-off makes sure the team builds software that aligns to the overall enterprise architecture.

Like any mature agile process, an agile approach to architecture relies on doing just enough definition up front to get started, gathering feedback as we go, adjusting as needed, and iterating frequently to keep architecture and design in sync with the emerging application.

Prior to the first sprint, a high-level enterprise or system architecture should be created (if it doesn’t already exist) and discussed. As part of each sprint kickoff, team-based design is used to update the as-is design from the last sprint to account for feature additions and enhancements that will be made in the current sprint. As stories are tasked out for development and testing, the team will use this new “to be” design to help determine what will need to be implemented in order to satisfy the expected design.

Sprint review meetings should present how each story fits into the bigger picture of the whole application and the enterprise. Sprint retrospectives should include architecture and design in the discussions of what is working and what isn’t.

In all facets, architecture should be treated like any other part of the agile process:

By starting with experienced architects and industry best practices, we can integrate the project requirements with the enterprise requirements and standards to be confident in a good foundation
By incorporating retrospection and review in the process, we can make minor adjustments to both the architecture and the software being developed to ensure that we meet the enterprise's needs
By driving that feedback using quality tools, security tools, testing, and other objective metrics, we ensure that we aren't reacting to guesses and anecdotal evidence
By reflecting and evaluating continuously, the course corrections remain minor and incremental and can adjust to changing architecture requirements throughout the life of the project and as needs evolve
By collaborating with the development teams, we can be sure the architecture reflects real-world needs and isn’t just an “ivory tower” architectural approach that makes sense on paper but fails in actual use.
For an organization transitioning to agile development, creating software architecture isn’t incompatible with your new processes. Consider the principles in the Agile Manifesto, involve team members who will be using the architecture in its development, and reflect and adapt often, and you will end up with an architecture that meets the needs of your team and your enterprise.