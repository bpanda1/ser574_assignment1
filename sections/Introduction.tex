Agile, now a days is the most trending and adaptable standard in software industry. Agile aims to deliver the software product incrementally from the starting of the project instead of delivering the whole product at once at the end. Teams following agile methodology collaborates with cross-functional disciplines to deliver. They work under operate within a common framework that guides for the end product. Each steps of software development like planning, integration, deployment and release of the product are done by the teams together. To work smoothly a simple, flexible, and an adaptable software design and architecture is necessary. 


Software architecture is the design and specification of the rules by which software will be built. It could be as high level as possible to give the overview of the project to extend the product. If the software has a lot of isolated components, then the architecture ensures that the system operates in an environment that supports components with isolated dependencies. It is more about collective understanding of how a system should be built to respond to the change.

Software Design act as a blueprint during the development. Having a correct and updated software design is must for the teams to work in agile method.
Software design may refer to either "all the activity involved in conceptualizing, framing, implementing, commissioning, and ultimately modifying complex systems" or "the activity following requirements specification and before programming, as ... [in] a stylized software engineering process [Citation: Freeman]. 
Software design focus on the detailed approach of software development. Grady Booch mentions Abstraction, Encapsulation, Modularization, and Hierarchy as fundamental software design principles in his object model.[4] The acronym PHAME (Principles of Hierarchy, Abstraction, Modularization, and Encapsulation) is sometimes used to refer to these four fundamental principles.

Section 2 describes the general idea of agile 
development. Section 3 focus on the primary steps involved in software design and architecture. Section 4 focuses on the impact of software design and architecture teams working in agile.



References:

Software Architecture — In an agile environment by Carlos Pliego [ https://medium.com/@carlospliego/agile-software-development-and-architecture-3dbb243fcbd2 ]

http://agilemodeling.com/essays/agileArchitecture.htm

Manifesto for Agile Software Development []

Freeman, Peter; David Hart (2004). "A Science of design for software-intensive systems". Communications of the ACM. 47 (8): 19–21 [20]. doi:10.1145/1012037.1012054.

Booch, Grady; et al. (2004). Object-Oriented Analysis and Design with Applications (3rd ed.). MA, USA: Addison Wesley. ISBN 0-201-89551-X. Retrieved 30 January 2015.

Suryanarayana, Girish (November 2014). Refactoring for Software Design Smells. Morgan Kaufmann. p. 258. ISBN 978-0128013977. Retrieved 31 January 2015.



