Architecture in agile has been defined in many different ways by many authors. It can be defined as a specific set of rules by which software will be built. It can also be defined as the design by which the components of a system will behave and interact. Architecture can be described in high level or detailed. High level meaning, it could just be specifies as the software is going to be built using APIs or detailed meaning giving the details of the technology stack of each and every component.

There are a number of reasons to explain why the architecture is important in Agile methodologies. Any software project involves a high cost and to mitigate the chances of increasing the cost, it is important for the software project to have a proper architectural design in its early stages. A specified architecture would not only give the form to a project but also acts as a preliminary test to see if the requirements of the customer have been properly incorporated by the development team. The architecture is also important in a way that without it, the project cost, schedule and quality cannot be predicted correctly. It also determines the key approach to choice of tools, work-breakdown, skills and technologies that are required to finish the project.

The architecture and design of a project is important in an agile environment. Hence it should be created in such a way as to accommodate the changes in the development phases such as [2]-
1. The requirements in any project are not stagnant and keep changing and evolving with respect to users’ needs, advancement in technology and also some constraints with respect to time, budget and labor.
2. These requirements change automatically leads to changes in software design and development too (code).
3. Due to the high budget of the project, accommodating all these changes would in turn lead to budget increase or cutting the cost of labor. This could be considered a expensive critical activity in rapidly changing requirements.
4. The architecture should also specify a clear software design process so as to avoid any miscommunication in terms of development which in turn might lead to bigger changes incurring higher costs.

Hence architecture should be designed in such a way that can take in all the changes, planned and unplanned, during the development of the software.

References:
https://www.scaledagileframework.com/agile-architecture/
Software Architectural design in Agile environments - Mehdi Mekni, Gayathri Buddhavarapu, Sandeep Chinthapatla, Mounika Gangula
